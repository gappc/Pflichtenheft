\section{Produktfunktionen}

\comment{Was leistet das Produkt aus Benutzersicht?}

\subsection{Benutzerfunktionen}

\subsubsection{Benutzer-Kennung}

Ein im System registrierter Benutzer kann das System erst nutzen, wenn er angemeldet ist.

\begin{description}
  \item[/F0010/]
    \textit{Registrieren:} Ein beliebiger Internet-Benutzer kann sich über die Start- bzw. Login-Seite des Systems
    schnell und bequem registrieren lassen. Zum Registrieren sind mindestens folgende Angaben erforderlich:
    \begin{itemize}
      \item gewünschte \textbf{Kennung}
        \begin{itemize}
          \item gewünschter \textbf{Benutzername}
          \item gewünschtes \textbf{Passwort}
        \end{itemize}
      \item eigene bzw. private \textbf{eMail-Adresse}
    \end{itemize}
    Die Registrierung ist erfolgreich, wenn der \textit{Benutzername} und die \textit{eMail-Adresse}
    innerhalb des Systems jeweils eindeutig sind. Die \textit{eMail-Adresse} wird auf ihre Gültigkeit geprüft.\\
    Mit dem erfolgreichen Abschießen des Registrierungsvorgangs ist der neue Benutzer am System angemeldet,
    zudem erhält der Benutzer automatisch via \textit{eMail} vom System seine aktuelle Kennung.
  \item[/F0020/]
    \textit{Anmelden:} Ein bereits registrierter Benutzer kann sich über die Start- bzw. Login-Seite des Systems
    schnell und bequem anmelden \textit{(login)}. Dazu ist seine Kennung erforderlich:
    \begin{itemize}
      \item sein \textbf{Benutzername}
      \item sein \textbf{Passwort}
    \end{itemize}
    Alternativ zum \textit{Benutzernamen} kann der Benutzer seine \textit{eMail-Adresse} angeben.
  \item[/F0030/]
    \textit{Abmelden:} Der angemeldete Benutzer kann sich jeder Zeit wieder vom System \textbf{abmelden} \textit{(logout)}.
  \item[/F0040/]
    \textit{Kennung anfordern:} Falls ein bereits registrierter Benutzer seine Kennung oder sein \textbf{Passwort vergessen}
    haben sollte, so kann er seine korrekte Kennung über die Start- bzw. Login-Seite des Systems anfordern.
    Dem Benutzer wird unter Angabe
    \begin{itemize}
      \item seines \textbf{Benutzernamens} oder
      \item seiner \textbf{eMail-Adresse}
    \end{itemize}
    seine vollständige Kennung automatisch via \textit{eMail} vom System zugesendet.
  \item[/F0050/]
    \textit{Passwort ändern:} Der angemeldete Benutzer kann das Passwort seiner Kennung ändern.
    Das neue Passwort muss zweimal angegeben werden, wobei sich diese Angaben nicht unterscheiden dürfen.
    Nach erfolgreicher Änderung des Passwortes erhält der Benutzer automatisch via \textit{eMail} vom System seine aktuelle Kennung.
\end{description}

Der Benutzer kann seinen \textit{Benutzernamen} nicht änderen.\\
\\
Im Folgenden sei der Benutzer stets am System angemeldet.

\subsubsection{Persönliche Daten}

Der Benutzer verfügt über persönliche Daten \textit{(siehe /D010/)}, die er frei gestalten kann.

\begin{description}
  \item[/F0110/]
    \textit{Anzeige der eigenen, persönlichen Daten:}
    Der Benutzer kann sich seine persönlichen Daten vom System \textbf{vollständig anzeigen} lassen.
  \item[/F0120/]
    \textit{Ändern der eigenen, persönlichen Daten:}
    Der Benutzer kann seine persönlichen Daten aktualisieren bzw. \textbf{ändern}.
  \item[/F0130/]
    \textit{Sichtbarkeit der eigenen, persönlichen Daten:}
    Der Benutzer kann jeden einzelnen Eintrag seiner persönlichen Daten für die Spielgemeinschaft auf \textbf{sichtbar} bzw.
    \textbf{unsichtbar} setzen.
  \item[/F0140/]
    \textit{Anzeige der persönlichen Daten anderer Benutzer:}
    Der Benutzer kann sich von anderen Benutzern die persönlichen Daten anzeigen lassen,
    dabei können auf unsichtbar gesetzte Einträge nicht gesehen werden.\\
    Im Gegensatz zu \textit{/F0110/} kann der Benutzer seine eigenen, persönlichen Daten auch auf diese Weise anzeigen lassen.
\end{description}


\subsubsection{Persönliche Konfiguration}

Die Nutzungsumgebung eines Benutzers ist das Layout, das Design, aber auch diverse logische Einstellungen,
die die individuelle Handhabung des Systems vereinfachen können.\\
Individuell einstellbar für den Benutzer sind:
\begin{itemize}
  \item die Farbgebung
  \item die Gliederung seiner Hauptseite \textit{(Anordnung von Menü, Spielfläche und Spieleliste)}
\end{itemize}
Zudem kann der Benutzer noch einstellen, welche Informationen direkt nach dem \textit{Login} auf der Hauptseite angezeigt werden sollen.
Persönliche Konfigurationen können verwaltet werden \textit{(siehe Portfolio)}.

\begin{description}
  \item[/F0210/]
    \textit{Anzeige der persönlichen Konfiguration:}
    Der Benutzer kann sich alle einstellbaren Werte seiner persönlichen Konfiguration seiner Nutzungsumgebung vom System \textbf{anzeigen} lassen.
  \item[/F0220/]
    \textit{Ändern der persönlichen Konfiguration:}
    Der Benutzer kann alle einstellbaren Werte seiner persönlichen Konfiguration \textbf{ändern}
    oder die voreingestellte Konfiguration wiederherstellen.
  \item[/F0230/]
    \textit{Speichern der persönlichen Konfiguration:}
    Der Benutzer kann seine persönliche Konfiguration \textbf{speichern} bzw. in seiner \textit{Portfolio} \textbf{sichern}.
  \item[/F0240/]
    \textit{Löschen der persönlichen Konfiguration:}
    Der Benutzer kann bereits gesicherte Konfigurationen aus seiner \textit{Portfolio} \textbf{entfernen}.
  \item[/F0250/]
    \textit{Wiederverwenden der persönlichen Konfiguration:}
    Der Benutzer kann bereits gesicherte Konfigurationen seiner \textit{Portfolio} \textbf{wiederverwenden}.
    Beim Wechseln der Konfiguration wird das gleichzeitige Sichern der aktuellen Konfiguration angeboten.
\end{description}

\subsubsection{Persönliches Profil}

Der Benutzer bzw. der Spieler verfügt über ein persönliches Profil.
Dieses kann man in zwei Teile gliederen:
\begin{itemize}
  \item allgemeines Profil
    \begin{itemize}
      \item die Häufigkeit des Erscheinens \textit{(Treue)}
    \end{itemize}
  \item Profil zu jedem Spieltyp \textit{(Mühle, Dame und Schach)}
    \begin{itemize}
      \item die Wertung \textit{(aktuelle Spielstärke in Form einer Zahl)}
      \item die Anzahl der gespielten, gewonnenen, verlorenen und unentschiedenen Spiele
      \item die Anzahl der noch offenen Spiele
      \item die Auflistung von Auszeichnungen
        \begin{itemize}
          \item der beste
          \item der schlechteste
          \item der schnellste
          \item der langsamste
          \item der treueste
          \item der bekannteste Spieler \textit{(der Woche, des Monats und des Jahres)}
        \end{itemize}
    \end{itemize}
\end{itemize}

Das persönliche Profil wird je nach Bedarf vom System automatisch aktualisiert, die Wertung wird z.B. nach dem Beenden einer Partie aktualisiert.

\begin{description}
  \item[/F0310/]
    \textit{Anzeige des eigenen, persönlichen Profils:}
    Der Benutzer kann sich sein persönliches Profil für jeden Spieltyp \textbf{anzeigen} lassen.
  \item[/F0320/]
    \textit{Sichtbarkeit des eigenen, persönlichen Profils:}
    Der Benutzer kann jeden einzelnen Eintrag seines persönlichen Profils für die Spielgemeinschaft auf \textbf{sichtbar} bzw.
    \textbf{unsichtbar} setzen.
    Jedoch die Wertungen bleiben immer öffentlich sichtbar.
  \item[/F0330/]
    \textit{Anzeige der persönlichen Profile anderer Benutzer:}
    Der Benutzer kann sich von anderen Benutzern die persönlichen Profile anzeigen lassen,
    dabei können auf unsichtbar gesetzte Einträge nicht gesehen werden.\\
    Im Gegensatz zu \textit{/F0310/} kann der Benutzer sein eigenes, persönliches Profil auch auf diese Weise anzeigen lassen.
\end{description}


\subsection{Spielfunktionen}

Der angemeldete Benutzer ist in erster Linie ein Spieler.
Dieser Spieler hat stets eine Liste von eigenen, laufenden Spielen zur Verfügung.\\
Ein Spieler kann nicht gegen sich selbst antreten.

\subsubsection{Initialisierung}

\begin{description}
  \item[/F0410/]
    \textit{Eröffnung eines Spieles:}
    Der angemeldete Benutzer kann Spiele \textbf{eröffnen}, ohne dabei einen anderen Spieler als Gegner angeben zu müssen.
    Ein eröffnetes Spiel kann von anderen Spieleren unter dem Menüpunkt \textbf{neue Spiele} angenommen werden \textit{/F0420/}.
  \item[/F0420/]
    \textit{Aufnahme eines Spieles:}
    Der angemeldete Benutzer kann bereits eröffnete Spiele \textbf{aufnehmen} \textit{/F0410/}.
  \item[/F0430/]
    \textit{Herausfordern eines Gegners:}
    Der angemeldete Benutzer kann unter Angabe eines gültigen Benutzernamens einen anderen Benutzer zum Spiel \textbf{herausfordern}.
  \item[/F0440/]
    \textit{Annahme einer Herausforderung:}
    Der angemeldete Benutzer kann eine Herausforderung zum Spiel \textit{/F0430/} \textbf{annehmen}.
  \item[/F0450/]
    \textit{Ablehnen einer Herausforderung:}
    Der angemeldete Benutzer kann eine Herausforderung zum Spiel \textit{/F0430/} \textbf{ablehnen}.
\end{description}

\subsubsection{Spielverlauf}

\begin{description}
  \item[/F0510/]
    \textit{Zugmöglichkeit:}
    Der angemeldete Benutzer kann zu jedem Spiel einen Zug seiner Wahl gemäß der Spielregeln \textbf{ziehen}, vorausgesetzt er ist am Zug.
  \item[/F0520/]
    \textit{Gebot eines Unentschieden:}
    Der angemeldete Benutzer kann gemäß den Spielregeln ein Unentschieden \textit{(remis)} \textbf{anbieten}, vorausgesetzt er ist am Zug.
  \item[/F0530/]
    \textit{Annahme eines Unentschieden:}
    Der angemeldete Benutzer kann, wenn ihm ein Unentschieden angeboten wurde \textit{/F0520/}, das Unentschieden \textbf{annehmen}.
  \item[/F0540/]
    \textit{Ablehnen eines Unentschieden:}
    Der angemeldete Benutzer kann, wenn ihm ein Unentschieden angeboten wurde \textit{/F0520/}, das Unentschieden \textbf{ablehnen}.
  \item[/F0550/]
    \textit{Aufgabe eines Spieles:}
    Der angemeldete Benutzer kann ein laufendes Spiel \textbf{mit Aufgabe beenden}, vorausgesetzt er ist am Zug.
  \item[/F0560/]
    \textit{Zugbedingter Nachrichtenaustausch:}
    Der angemeldete Benutzer kann zu jedem Zug \textbf{eine Nachricht übermitteln}.
\end{description}

\subsection{Portfolio-Funktionen}

Der Benutzer verfügt über eine persönliche Sammel- und Dokumentenmappe,
die mit Such-, Sortier- und Dokumentationsfunktionen versehen ist.
Dieses Portfolio ist rein zur privaten Verwendung und deshalb nicht für die Spielgemeinschaft sichtbar.
Das Portfolio ermöglicht das individuelle Dokumentieren und Sammeln im System.\\
Der Benutzer kann grundsätzlich alles Sammelbare in sein Portfolio aufnehmen.\\
Zum Sammelbaren zählen:
\begin{itemize}
  \item persönliche Konfigurationen
  \item andere Benutzer bzw. Spieler
  \item bereits beendete Spiele
  \item Nachrichten
  \item Notizen
\end{itemize}

\begin{description}
  \item[/F0610/]
    \textit{Anzeige des Portfolios:}
    Der Benutzer kann sich den Inhalt des persönlichen Portfolios \textbf{anzeigen} lassen.
  \item[/F0620/]
    \textit{Suche nach Benutzern:}
    Der Benutzer kann mit der Suchfunktion des Portfolios nach anderen Benutzern des Systems anhand einer beliebigen Kombination
    der folgenden Kriterien suchen:
    \begin{itemize}
      \item Benutzername
      \item Bereichsangabe der Wertungen zu gegebenen Spieltyp
    \end{itemize}
  \item[/F0630/]
    \textit{Suche nach Spielen:}
    Der Benutzer kann mit der Suchfunktion des Portfolios nach bereits gespielten Spielen des Systems anhand einer beliebigen Kombination
    der folgenden Kriterien suchen:
    \begin{itemize}
      \item Spieltyp \textit{(Mühle, Dame, Schach oder alle)}
      \item Anzahl der Spielzüge
      \item Gewinnfarbe \textit{(Weiss, Schwarz oder unentschieden)}
    \end{itemize}
\end{description}

\subsection{Administratorfunktionen}

Der Administrator verfügt über alle Benutzerfunktionen, und kann darüberhinaus die Eigenschaften des Systems konfigurieren.
Zudem kann der Administrator Benutzer aus dem System verbannen,
sowie den Informationsaustausch \textit{(Instant-Messaging)} zwischen zwei Benutzern völlig unterbinden,
sofern diese kein Spiel am Laufen haben.

\subsubsection{Systemverwaltung}

\begin{description}
  \item[/F1010/]
    \textit{Konfiguration:}
    Der angemeldete Administrator kann die Eigenschaften des Systems
    \begin{itemize}
      \item Sessiondauer eines Benutzers \textit{(Autologout)}
      \item Welche Werbebanner auf welchen Seiten angezeigt werden sollen
    \end{itemize}
    konfigurieren.
  \item[/F1020/]
    \textit{Statistiken:}
    Der angemeldete Administrator kann sich Statistiken
    \begin{itemize}
      \item Welcher Spieltyp wird am häufigsten verwendet
      \item Wieviele Benutzer registrieren sich pro Tag
      \item Wieviele Benutzer sind bzw. waren am System angemeldet
    \end{itemize}
    zur Benutzung des Systems anzeigen lassen.
\end{description}

\subsubsection{Benutzerverwaltung}

\begin{description}
  \item[/F1110/]
    \textit{Einschänkung der Benutzer:}
    Der angemeldete Administrator kann die Eigenschaften einzelner Benutzer unter Angabe einer zeitlichen Begrenzung einschränken.
    \begin{itemize}
      \item Er kann die Möglichkeit zum Nachrichtenaustausch \textit{(Instant-Messaging)} zweier Benutzer unterbinden.
      \item Er kann jeglichen Kontakt zwischen zwei Benutzer unterbinden, wobei bereits eröffnete Spiele zu Ende gespielt werden müssen.
    \end{itemize}
    Eine völlige Einschränkung eines Benutzers bedeutet die Verbannung eines Benutzers aus dem System.
    Dies ist besonders sinnvoll, wenn ein Benutzer bei einem kostenpflichtigen Account seinen Zahlungen nicht nachkommen sollte
    \textit{(Ausblick auf Folgeversion)}.
  \item[/F1120/]
    \textit{Einschränkungen restaurieren:}
    Der angemeldete Administrator kann die Eigenschaften einzelner Benutzer auch wieder manuell restaurieren.
\end{description}
