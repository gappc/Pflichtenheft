\section{Zielbestimmungen}

\comment{Welche Musskriterien, Wunschkriterien, Abgrenzungskriterien sind erforderlich?}

\textbf{Brettspiele} stellt einen Internetdienst dar, der das Spielen von Brettspielen zwischen Einzelpersonen ermöglicht.
Im Folgenden bezeichne \textit{Benutzer} und \textit{Spieler} sowohl die weiblichen, als auch die männlichen Benutzer und Spieler.

\subsection{Musskriterien}

\subsubsection{Kernfunktionalität}

\subsubsection{Der Benutzer-Account}
\begin{itemize}
	\item Der Internet-Benutzer kann sich selbst am System registrieren.
	\item Der Benutzer kann sich am System anmelden und vom System abmelden.
	\item Der Benutzer kann seine Kennung anfordern.
	\item Der Benutzer kann seine persönlichen Daten einsehen und ändern, sowie deren Sichtbarkeit innerhalb der Spielgemeinschaft einstellen.
	\item Der Benutzer kann die persönlichen Daten anderer Benutzer einsehen, soweit diese sichtbar sind.
	\item Der Benutzer kann sein persönliches Profil einsehen, sowie die Sichtbarkeit innerhalb der Spielgemeinschaft einstellen.
	\item Der Benutzer kann das persönliche Profil anderer Benutzer einsehen, soweit dieses sichtbar ist.
	\item Der Benutzer kann seine Nutzungsoberfläche nach eigenem Bedarf und Geschmack konfigurieren.
	\item Der Benutzer kann Konfigurationen der eigenen Nutzungsoberfläche neu erstellen, speichern, löschen, ändern und wieder verwenden.
	\item Der Benutzer ist in Besitz einer eigenen Portfolio, in der er Konfigurationen, Benutzer, Spiele, Nachrichten und Notizen verwalten kann.
	\item Der Benutzer kann mit den Funktionen des Portfolios das System durchsuchen, die Suchergebnisse können dem Portfolio hinzugefügt werden.
	\item Der Benutzer kann jeden einzelnen Eintrag im Portfolio kommentieren und bewerten.
	\item Die Benutzer können untereinander Nachrichten austauschen (Instant-Messaging).
\end{itemize}

\subsubsection{Das Spiel}
\begin{itemize}
	\item Es stehen drei Spieltypen zur Verfügung: \textit{Mühle}, \textit{Dame} und \textit{Schach}.
	\item Der Spieler kann Spiele beliebigen Typs eröffnen.
	\item Der Spieler kann ein eröffnetes Spiel aufnehmen.
	\item Die Spieler können sich untereinander zum Spiel herausfordern.
	\item Der Spieler muss gemäß den Spielregeln ziehen, ein Unentschieden anbieten/annehmen oder aufgeben bzw. gewinnen.
	\item Der Spieler kann bei jeder Spielaktion eine kurze Nachricht übermitteln.
	\item Nach dem Beenden einer Partie wird das persönliche Profil gemäß der Spielstärke des Spielers mit einer zum Spieltyp gehörenden Berechnungsfunktion aktualisiert.
\end{itemize}

\subsubsection{Der Administrator}
\begin{itemize}
	\item Der Administrator konfiguriert die Betriebsparameter des Systems.
	      %\item Der Administrator muss den Zugang aller Benutzer vorübergehend sperren können,
	\item Der Administrator sichert die Datenbank.
\end{itemize}
\subsubsection{Sonstiges}
\begin{itemize}
	\item Englisch als Verkehrssprache.
	\item Erweiterbarkeit des Systems weiterer europäischen Sprachen.
	\item Erweiterbarkeit des Systems weiterer Spieltypen \textit{(z.B. Go oder Backgammon)}.
	\item Erweiterbarkeit des Systems weiterer Bereiche \textit{(z.B. Spiele-Forum, Lehrbücherverkauf)}.
	\item Erweiterbarkeit des Systems zur gebührenpflichten Nutzung für die Benutzer.
\end{itemize}

\subsection{Wunschkriterien}

\begin{itemize}
	\item Die Benutzer können die Einträge im Portfolio untereinander austauschen.
	\item Der Benutzer kann den Verlauf seiner beendeten Spiele einsehen.
	\item Zu jedem Spieltyp liegen die Spielregeln im System vor.
	\item Einfache Integration von Werbebannern, einstellbar durch Administrator.
\end{itemize}

\subsection{Abgrenzungskriterien}

\begin{itemize}
	\item Nur Fernspiele, also z.B. Fernschach und kein Blitzschach.
	\item Das System eignet sich nur für Zwei-Spieler-Spiele.
\end{itemize}
