\section{Produkteinsatz}

\comment{Welche Anwendungsbereiche (Zweck), Zielgruppen (Wer mit welchen Qualifikationen), Betriebsbedingungen (Betriebszeit, Aufsicht)?}

\subsection{Anwendungsbereiche}

Einzelpersonen verwenden diesen Dienst zum Spielen der oben angegebenen Brettspiele mit anderen Personen der Spielgemeinschaft.
Diese Plattform soll dem Einzelnen eine Kommunikation mit Gleichgesinnten ermöglichen, um so ihre Fertigkeiten im Spiel verbessern zu können.

\subsection{Zielgruppen}

Personengruppen, die kurz zur Ablenkung z.B. in der Mittagspause, gerne an Fernspielen teilhaben,
in dem sie sich Gedanken über bevorstehende Spielzüge machen können.

Diese Plattform ist für Einzelpersonen gedacht, die in ihrer knapp bemessenen Freizeit Schwierigkeiten haben,
ihrem Hobby z.B. Schach nachzugehen oder Gegner zu finden.\\
\\
Es werden Basiskenntnisse in Internetnutzung vorausgesetzt. Ebenso die Spielregeln des jeweiligen Spieltyps sollten vor der
Nutzung bekannt sein.\\
\\
Soweit keine weiteren Sprachen integriert sind, muss der Benutzer die Verkehrssprache \textit{Englisch} zumindest verstehen.

\subsection{Betriebsbedingungen}

Dieses System soll sich bezüglich der Betriebsbedingungen nicht wesentlich von anderen Internetdiensten bzw. -anwendungen unterscheiden.

\begin{itemize}
	\item Betriebsdauer: täglich, 24 Stunden
	\item Wartungsfrei
	\item Die Sicherung der Datenbank muss manuell vom Administrator durchgeführt werden.
	\item Falls nötig, ist der Administrator zur Schlichtung zwischen Benutzern verantwortlich.
\end{itemize}
