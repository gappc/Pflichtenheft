\section{Benutzungsoberfl�che}

\comment{Was sind die grundlegenden Anforderungen an die Benutzungsoberfl�che (Bildschirmlayout, Dialogstruktur, ...)?}

\subsection{Dialogstruktur}

Im Folgenden wird die grobe Dialogstruktur einer fehlerfreien bzw. konfliktfreien Benutzung des Systems gezeigt.
Fehlereingaben haben in der Regel einen R�cksprung auf die Ausgangsseite mit einer akkumulierten Fehlermeldung zur Folge.

\subsubsection{Startseite}

\spaceline
\btexdraw
  \rmove(3 3)
   \xstartpage \xhline \currentpos \sx\sy
    \rlvec(0 \yheightb) \xharrow \xtext{4.5}{Registrierung \textbf{/F0010/}} \xhline \currentpos \jx\jy
      \rlvec(0 \yheightd) \xharrow \xmainpage \xnext
      \move({\jx} \jy)
      \rlvec(0 -\yheightd) \xharrow \xaccount
    \move({\sx} \sy)
    \rlvec(0 -\yheightd) \currentpos \px\py \xharrow \xtext{4.1}{Anmeldung \textbf{/F0020/}} \xharrow \xmainpage \xnext
    \move({\px} \py)
    \rlvec(0 -\yheightb) \xharrow \xtext{5.5}{Passwort vergessen! \textbf{/F0040/}} \xharrow \xaccount
\etexdraw
\spaceline

\subsubsection{Hauptseite}

Die \textit{Hauptseite} ist die Startseite des angemeldeten Benutzers, die der Benutzer gem�� der Funktion \textit{/F0220/}
konfigurieren kann.\\
Unabh�ngig von der \textit{pers�nlichen Konfiguration der Hauptseite} liegt folgende Dialogstruktur vor.

\spaceline
\btexdraw
  \rmove(3 3)
  \xmainpage \xhline \currentpos \sx\sy
    \rlvec(0 \yheighte) \currentpos \mx\my
    \rlvec(0 \yheightb) \xharrow \xusermenu \xnext
    \move({\mx} \my)
    \xharrow \xgamelist \xnext
    \move({\sx} \sy)
    \rlvec(0 -\yheighte) \currentpos \nx\ny \xharrow \xportfolio \xnext
    \move({\nx} \ny)
    \rlvec(0 -\yheightb) \xharrow \xtext{4.1}{Abmeldung \textbf{/F0030/}} \xharrow \xstartpage \xnext
\etexdraw
\spaceline

\subsubsection{Benutzermen�}

\spaceline
\btexdraw
  \rmove(7 7)
  \xusermenu \xhline \currentpos \sx\sy \xharrow \xtext{6.8}{Pers. Konfiguration anzeigen \textbf{/F0210/}}
    \move({\sx} \sy) \rlvec(0 \yheightb) \currentpos \mx\my \xharrow \xtext{4.8}{Passwort �ndern \textbf{/F0050/}} \xharrow \xaccount
    \move({\mx} \my) \rlvec(0 \yheightb) \currentpos \nx\ny \xharrow \xtext{6.7}{Sichtbarkeit der pers. Daten \textbf{/F0130/}}
    \move({\nx} \ny) \rlvec(0 \yheightb) \currentpos \ox\oy \xharrow \xtext{5.3}{Pers. Daten �ndern \textbf{/F0120/}}
    \move({\ox} \oy) \rlvec(0 \yheightb) \xharrow \xtext{5.6}{Pers. Daten anzeigen \textbf{/F0110/}}
    
    \move({\sx} \sy) \rlvec(0 -\yheightb) \currentpos \ax\ay \xharrow \xtext{6.5}{Pers. Konfiguration �ndern \textbf{/F0220/}}
    \move({\ax} \ay) \rlvec(0 -\yheightb) \currentpos \bx\by \xharrow \xtext{6.9}{Pers. Konfiguration speichern \textbf{/F0230/}}
    \move({\bx} \by) \rlvec(0 -\yheightb) \currentpos \cx\cy \xharrow \xtext{5.5}{Pers. Profil anzeigen \textbf{/F0310/}}
    \move({\cx} \cy) \rlvec(0 -\yheightb) \xharrow \xtext{6.9}{Sichtbarkeit des pers. Profils \textbf{/F0320/}}
\etexdraw
\spaceline

\subsection{Bildschirmlayout}

Das Layout sowie das Design des Systems wird �berwiegend durch JavaScript-Komponenten der Bibliothek \textit{dynAPI} bestimmt
und ist �ber das gesamte System konsistent bzw. einheitlich \textit{(Ausnahme: die Administrator-Funktionen)}.

